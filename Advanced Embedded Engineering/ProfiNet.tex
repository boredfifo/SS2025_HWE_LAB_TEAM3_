\documentclass[conference]{IEEEtran}
\IEEEoverridecommandlockouts
% The preceding line is only needed to identify funding in the first footnote. If that is unneeded, please comment it out.
\usepackage{cite}
\usepackage{amsmath,amssymb,amsfonts}
\usepackage{algorithmic}
\usepackage{graphicx}
\usepackage{textcomp}
\usepackage{xcolor}
\def\BibTeX{{\rm B\kern-.05em{\sc i\kern-.025em b}\kern-.08em
    T\kern-.1667em\lower.7ex\hbox{E}\kern-.125emX}}
\begin{document}

\title{ProfiNet\\
{\footnotesize \textsuperscript{*}Note: Sub-titles are not captured in Xplore and
should not be used}
\thanks{Identify applicable funding agency here. If none, delete this.}
}

\author{\IEEEauthorblockN{1\textsuperscript{st} Rubayet Kamal}
\IEEEauthorblockA{\textit{Electronic Engineering} \\
\textit{Hochschule Hamm-Lippstadt}\\
Lippstadt, Germany\\
rubayet.kamal@stud.hshl.de}
\and
\IEEEauthorblockN{2\textsuperscript{nd} Moiz Zaheer Malik}
\IEEEauthorblockA{\textit{Electronic Engineering} \\
\textit{Hochschule Hamm-Lippstadt}\\
Lippstadt, Germany\\
moiz-zaheer.malik@stud.hshl.de}
\and
\IEEEauthorblockN{3\textsuperscript{rd} Mofifolowa Ipadeola Akinwande}
\IEEEauthorblockA{\textit{Electronic Engineering} \\
\textit{Hochschule Hamm-Lippstadt}\\
Lippstadt, Germany\\
mofifoluwa-ipadeola.akinwande@stud.hshl.de}
}

\maketitle

\begin{abstract}
Real-time communication between devices and systems in industrial automation is crucial. In the past, industries relied on fieldbus communications and serial communications to achieve this. However, the integration of Ethernet-based protocols to these communications served as a turning point as they offered more potential, like higher bandwidth, faster switching technology, and easier access. Among these protocols, PROFINET, the successor of PROFIBUS, has emerged as one of the most widely used protocols in the industry, enabling efficient, scalable, and deterministic data exchange. In this paper, we explore the capabilities of PROFINET, compare it to other existing protocols, and highlight why it is the best industrial communication protocol.

\end{abstract}

\begin{IEEEkeywords}
component, formatting, style, styling, insert
\end{IEEEkeywords}

\section{Introduction}
This document is a model and instructions for \LaTeX.
Please observe the conference page limits. 

\section{Application Areas}
\section{ProfiSAFE}
\section{Comparison}
\section{Conclusion}




\end{document}
